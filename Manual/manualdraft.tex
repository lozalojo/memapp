\documentclass[10pt,a4paper]{memoir}
\usepackage{natbib}

\usepackage{lmodern}
\usepackage[T1]{fontenc}
\usepackage[english,activeacute]{babel}
\usepackage{mathtools}
\usepackage[utf8]{inputenc}
\usepackage[english]{isodate}
\usepackage{listings}
\usepackage{color}
\lstset{ %
  language=R,                     % the language of the code
  basicstyle=\footnotesize,       % the size of the fonts that are used for the code
  numbers=left,                   % where to put the line-numbers
  numberstyle=\tiny\color{gray},  % the style that is used for the line-numbers
  stepnumber=1,                   % the step between two line-numbers. If it's 1, each line
                                  % will be numbered
  numbersep=5pt,                  % how far the line-numbers are from the code
  backgroundcolor=\color{white},  % choose the background color. You must add \usepackage{color}
  showspaces=false,               % show spaces adding particular underscores
  showstringspaces=false,         % underline spaces within strings
  showtabs=false,                 % show tabs within strings adding particular underscores
  frame=single,                   % adds a frame around the code
  rulecolor=\color{black},        % if not set, the frame-color may be changed on line-breaks within not-black text (e.g. commens (green here))
  tabsize=2,                      % sets default tabsize to 2 spaces
  captionpos=b,                   % sets the caption-position to bottom
  breaklines=true,                % sets automatic line breaking
  breakatwhitespace=false,        % sets if automatic breaks should only happen at whitespace
  title=\lstname,                 % show the filename of files included with \lstinputlisting;
                                  % also try caption instead of title
  keywordstyle=\color{blue},      % keyword style
  commentstyle=\color{dkgreen},   % comment style
  stringstyle=\color{mauve},      % string literal style
  escapeinside={\%*}{*)},         % if you want to add a comment within your code
  morekeywords={*,...}            % if you want to add more keywords to the set
} 

\title{%
  The Moving Epidemics Method \\
  \large The mem Shiny web application}
\author{José E. Lozano and  Jakob Bergström}
\date{\today}

\begin{document}
% cuerpo del documento

\maketitle
\newpage
% \pagenumbering{Roman}
\tableofcontents
% \newpage
% \listoffigures
% \newpage
% \listoftables
\newpage
\pagenumbering{arabic}

\chapter{Introduction}

The Moving Epidemics Method (MEM) is a tool developed in Castilla y León (Spain) to help in the routine influenza surveillance in health systems. It gives a better understanding of the annual influenza epidemics and allows the weekly assessment of the epidemic status and intensity.

Thought in its conception it was originally created to be used with influenza data and health sentinel networks, MEM has been tested with different diseases and surveillance systems so nowadays it can be used with any disease which present a seasonal accumulation of cases that can be considered an epidemic.

MEM development started in 2000 and the first record of is existence is dated in 2003 in the Options for the Control of Influenza V\citep{vega_alonso_modelling_2004}.

It was presented to the baselines working group of the European Influenza Surveillance Scheme (EISS) in the 12th EISS Annual Meeting (Malaga, Spain, 2007), with whom started a collaboration that continued when EISS was dissolved in 2008 to create the European Influenza Surveillance Network.

In 2009 MEM appears for the first time in an official European document: the Who European guidance for influenza surveillance in humans. A year later MEM was implemented in the European Centre for Disease Prevention and Control (ECDC) platform, and in 2012, after piloting, in the World Health Organization Regional Office for Europe (WHO-E).

As a result of the collaboration with ECDC and WHO-E, two papers have been published, one related to the establishment of epidemic thresholds\citep{vega_influenza_2013} and other in the comparison of intensity levels in Europe\citep{vega_influenza_2015}.

In 2014 a tool was created to help users around the world to apply mem on their data. It was released in July as a library for R, a free software environment for statistical computing and graphics. It is available at the official repositories: The Comprehensive R Archive Network (CRAN), it is the stable mem version\citep{jose_e_lozano_alonso_mem_nodate}.
In 2015 the second version of the mem R library was published open source at GitHub, a web-based Git or version control repository and Internet hosting service. This is available directly from github\citep{lozano_jose_e_lozalojo/mem:_nodate} and is considered as the development version and includes a lot of new features and graphics.

In 2017 a web application was created to serve as a graphical user interface for the R mem library using a web application framework for R called Shiny. This application is based on the development version of the mem R library.

\chapter{Installation}

The mem Shiny web application (memshy) is based on the mem R library and requires R to work. R is available as Free Software under the terms of the Free Software Foundation’s GNU General Public License in source code form. It compiles and runs on a wide variety of UNIX platforms and similar systems (including FreeBSD and Linux), Windows and MacOS. There are binaries for most operating systems at its official web page\citep{the_r_foundation_r_nodate}. To install download the binaries appropriate for your system and proceed to install it.

R is a command line program but there are a lot of graphical user interfaces available to users that wants a friendlier environment. The most popular is RStudio an open source powerful and productive user interface for R. Binaries can be downloaded from its official web\citep{rstudio_r_nodate} and installed on Windows, Mac, and Linux.

Shiny is a web application framework for R created by the RStudio team, there is no need to install separately because it acts as a library for the R language and will be installed with the rest of dependencies.

MEM Shiny app is a set of two files that Shiny framework is able to interpret in order to start the web application. They can be run directly from a remote web server or in a local directory of our hard disc (running a local server).

\section{Dependencies}

Packages are collections of R functions, data, and compiled code in a well-defined format. The directory where packages are stored is called the library. R is based on the contribution of the community, which creates their own code and share it creating new packages, which extend the features of the base package.

Most packages has requirements, they need other packages to work, these are the dependencies of the package.

MEM Shiny app and mem R library requires to install a set of packages (dependencies) on R to start the application. The list of dependencies required for both applications are those of the original mem R library and those added by the new features of the mem Shiny web application.

Mem R library requirements:

\begin{itemize}[\textbullet]
\item sm
\item boot
\item grDevices
\item graphics
\item stats
\item sqldf
\item reshape2
\item RColorBrewer
\item mixtools
\end{itemize}

memshy requeriments:

\begin{itemize}[\textbullet]
\item shiny
\item shinythemes
\item shinydashboard
\item shinyjs
\item shinyBS
\item plotly
\item ggplot2
\item ggthemes
\item R.utils
\item openxlsx
\item XLConnect
\item stringr
\item readr
\item magick
\item DT
\item gplots
\item RODBC
\item mem
\item shinysky
\end{itemize}

Almost all the libraries can be installed directly from the CRAN repositories. To install a package simply write in the command line:

\begin{lstlisting}
install.package('packagename')
\end{lstlisting}

Installation of packages must be done only once, after a package is installed, to use it, it has to be loaded, no need to install it again.

\begin{lstlisting}
library('packagename')
\end{lstlisting}

To automatize the work of installing and loading libraries, here it is a custom function to check all dependencies and install in case it is needed for memshy.

\begin{lstlisting}
testinstall.packages <- function(i.packages) {
  lapply(i.packages, function (x) if(sum(installed.packages()[, 1]%in%x)==0) install.packages(x))
  lapply(i.packages, require, character.only=TRUE)
  lapply(i.packages,  function(x) paste(x, packageVersion(x)))
}
testinstall.packages(c('shiny', 'shinythemes', 'shinydashboard', 'shinyjs', 'RColorBrewer', 'shinyBS',
              'plotly', 'ggplot2', 'ggthemes', 'reshape2', 'R.utils', 'openxlsx', 'XLConnect',
              'stringr', 'readr', 'magick', 'DT', 'gplots', 'RODBC', 'mixtools'))
\end{lstlisting}

There are two missing packages in the above code, requires special installation, the mem R library is hosted at github, and the shinysky package does not provide binaries to the lastest version of R, so it has to be installed from sources, also at github.

To install from github another package must be installed: devtools.

\begin{lstlisting}
install.package('devtools')
\end{lstlisting}

And to install and load the last two packages:

\begin{lstlisting}
# Install mem development version
if ('mem' %in% installed.packages()[,'Package']){
  if (as.numeric(as.character(packageVersion('mem')))<2){
    testinstall.packages('devtools')
    devtools::install_github('lozalojo/mem')
  }  
}else{
  testinstall.packages('devtools')
  devtools::install_github('lozalojo/mem')
}
library('mem')
# There are no binaries for shinysky
if (!('shinysky' %in% installed.packages()[,'Package'])) devtools::install_github('AnalytixWare/ShinySky')
library('shinysky')
\end{lstlisting}

In windows, the only package that can give problems installing is rJava (required by XLConnect), which requires Java installed in your computer (www.java.com).

In *nix systems, there are libraries that need some specific packages installed on your system, each system is different, but these are some issues we’ve found while installing on debian and arch linux machines:

\begin{itemize}
	\item hexbin (required by plotly) requires a fortran compiler.
	\item rJava (required by XLConnect) requires Java.
	\item RODBC requires an odbc package (unixodbc).
\end{itemize}

Finally, memshy is able to read access files. In *nix systems it is required to install mdbtools (http://mdbtools.sourceforge.net/), which is probably in the distribution repositories.

Once all requirements are fulfilled, you can start the application.

\chapter{Running memshy}

From RStudio, to run a Shiny application is easy:

\begin{lstlisting}
runApp(path.to.the.files, launch.browser = T)
\end{lstlisting}














\medskip
 
\bibliographystyle{unsrt}
\bibliography{manualdraft}

\end{document}
